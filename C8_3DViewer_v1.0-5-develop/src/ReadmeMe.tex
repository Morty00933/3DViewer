\documentclass[12pt, a4paper]{article}

\usepackage[utf8]{inputenc}
\usepackage{graphicx}
\usepackage[russian]{babel}

\title{3D-Viewer on C}

\authors{florentq kaycekey charolet}

\date{September 2023}


\begin{document}

% first manual page
\maketitle

\pagebreak

\tableofcontents

\pagebreak

\section{Description}

Проект SmartCalc v1.0 - калькулятор с расширенным функционалом построения графиков.

Калькулятор написан на языке программирования Си. Поддерживает ввод данных через интерфейс. В калькуляторе представлены 2 вида х (Х, х). Первый подходит для построения графика, второй для подставления х в уравнение, для его вычисления.

\pagebreak

\section{Функции для визуализации каркасной модели в трехмерном пространстве}

Приложение 3D Viewer предоставляет возможность :
  \begin{itemize}
    \item Функции
    \begin{itemize}
      \item  Загружать каркасную модель из файла формата obj.
      \item Перемещать модель на заданное расстояние относительно осей X, Y, Z.
      \item Поворачивать модель на заданный угол относительно своих осей X, Y, Z.
      \item Масштабировать модель на заданное значение.
    \end{itemize}
  \end{itemize}

  Программа позволяет настраивать тип проекции (параллельная и центральная), тип (сплошная, пунктирная), цвет и толщину ребер, способ отображения (отсутствует, круг, квадрат), цвет и размер вершин, позволяет выбирать цвет фона.
Также программа позволяет сохранять полученные ("отрендеренные") изображения в файл в форматах bmp и jpeg и записывать небольшие "скринкасты" в gif-анимацию. Настройки программы могут сохраняться между перезапусками для удобства пользователей.

\pagebreak
